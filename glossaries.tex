%Glossary
\newglossaryentry{dot}{type=\glsdefaulttype, name=DOT, description={è un linguaggio di descrizione del grafico. I grafici DOT sono in genere file con l'estensione del nome file \texttt{.gv} o \texttt{.dot}. Normalmente viene utilizzata l'estensione \texttt{.gv}, per evitare confusione con l'estensione \texttt{.dot} utilizzata dalle versioni di \gls{word} prima del 2007. \cite{wikipedia}}}

\newglossaryentry{word}{name=Microsoft Word, description={è un elaboratore di testi pubblicato da Microsoft. È una delle applicazioni per la produttività dell'ufficio incluse nella suite Microsoft Office \cite{wikipedia}}}

\newglossaryentry{hashicorp}{name=HashiCorp, description={è una società di software con sede a San Francisco, California. HashiCorp fornisce strumenti open source e prodotti commerciali che consentono a sviluppatori, operatori e professionisti della sicurezza di fornire, proteggere, eseguire e connettere l'infrastruttura di cloud computing. È stata fondata nel 2012 da Mitchell Hashimoto e Armon Dadgar \cite{wikipedia}}}

\newglossaryentry{go}{type=\glsdefaulttype, name=Go, description={è un linguaggio di programmazione open source sviluppato da Google. Il lavoro su Go nacque nel settembre 2007 da Robert Griesemer, Rob Pike e Ken Thompson \cite{wikipedia}}}

%Glossary Acronyms entry
\newglossaryentry{iacg}{type=\glsdefaulttype, name=Infrastructure as Code, description={è il processo di gestione e provisioning dei data center e computer tramite file di definizione leggibili, piuttosto che un processo di configurazione hardware fisico o tramite strumenti di configurazione interattivi  \cite{wikipedia}}}

\newglossaryentry{awsg}{type=\glsdefaulttype, name=Amazon Web Services, description={ è un'azienda statunitense di proprietà del gruppo Amazon, che fornisce servizi di cloud computing \cite{wikipedia}}}

\newglossaryentry{sscg}{name=Strongly connected component, description={un grafo orientato è fortemente connesso se esiste un percorso tra tutte le coppie di vertici  \cite{wikipedia}}}

\newglossaryentry{dfsg}{name=Deep-First Search, description={è un algoritmo di ricerca su alberi e grafi. Il nome deriva dal fatto che in un albero, ancora prima di avere visitato i nodi vicino alla radice, l'algoritmo può ritrovarsi a visitare vertici lontani dalla radice, andando così "in profondità" \cite{wikipedia}}}

%Acronyms
\newglossaryentry{hcl}{type=\acronymtype, name={HCL}, description={HashiCorp Configuration Language}}

\newglossaryentry{ami}{type=\acronymtype, name={AMI}, description={Amazon Machine Images}}

%Acronyms glossary entry
\newglossaryentry{iac}{type=\acronymtype, name={IaC}, description={Infrastructure as Code}, first={Infrastructure as Code (IaC)\glsadd{iacg}}, see=[Glossary:]{iacg}}

\newglossaryentry{aws}{type=\acronymtype, name={AWS}, description={Amazon Web Services}, first={Amazon Web Services (AWS)\glsadd{awsg}}, see=[Glossary:]{awsg}}

\newglossaryentry{ssc}{type=\acronymtype, name={SSC}, description={Strongly connected component}, first={Strongly connected component  (SSC)\glsadd{sscg}}, see=[Glossary:]{awsg}}

\newglossaryentry{dfs}{type=\acronymtype, name={DFS}, description={Deep-First Search}, first={Deep-First Search  (DFS)\glsadd{dfsg}}, see=[Glossary:]{awsg}}